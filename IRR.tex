
%% bare_jrnl.tex
%% V1.3
%% 2007/01/11
%% by Michael Shell
%% see http://www.michaelshell.org/
%% for current contact information.
%%
%% This is a skeleton file demonstrating the use of IEEEtran.cls
%% (requires IEEEtran.cls version 1.7 or later) with an IEEE journal paper.
%%
%% Support sites:
%% http://www.michaelshell.org/tex/ieeetran/
%% http://www.ctan.org/tex-archive/macros/latex/contrib/IEEEtran/
%% and
%% http://www.ieee.org/

% !TeX spellcheck = en_US


% *** Authors should verify (and, if needed, correct) their LaTeX system  ***
% *** with the testflow diagnostic prior to trusting their LaTeX platform ***
% *** with production work. IEEE's font choices can trigger bugs that do  ***
% *** not appear when using other class files.                            ***
% The testflow support page is at:
% http://www.michaelshell.org/tex/testflow/


%%*************************************************************************
%% Legal Notice:
%% This code is offered as-is without any warranty either expressed or
%% implied; without even the implied warranty of MERCHANTABILITY or
%% FITNESS FOR A PARTICULAR PURPOSE! 
%% User assumes all risk.
%% In no event shall IEEE or any contributor to this code be liable for
%% any damages or losses, including, but not limited to, incidental,
%% consequential, or any other damages, resulting from the use or misuse
%% of any information contained here.
%%
%% All comments are the opinions of their respective authors and are not
%% necessarily endorsed by the IEEE.
%%
%% This work is distributed under the LaTeX Project Public License (LPPL)
%% ( http://www.latex-project.org/ ) version 1.3, and may be freely used,
%% distributed and modified. A copy of the LPPL, version 1.3, is included
%% in the base LaTeX documentation of all distributions of LaTeX released
%% 2003/12/01 or later.
%% Retain all contribution notices and credits.
%% ** Modified files should be clearly indicated as such, including  **
%% ** renaming them and changing author support contact information. **
%%
%% File list of work: IEEEtran.cls, IEEEtran_HOWTO.pdf, bare_adv.tex,
%%                    bare_conf.tex, bare_jrnl.tex, bare_jrnl_compsoc.tex
%%*************************************************************************

% Note that the a4paper option is mainly intended so that authors in
% countries using A4 can easily print to A4 and see how their papers will
% look in print - the typesetting of the document will not typically be
% affected with changes in paper size (but the bottom and side margins will).
% Use the testflow package mentioned above to verify correct handling of
% both paper sizes by the user's LaTeX system.
%
% Also note that the "draftcls" or "draftclsnofoot", not "draft", option
% should be used if it is desired that the figures are to be displayed in
% draft mode.
%
\documentclass[journal]{IEEEtran}
%
% If IEEEtran.cls has not been installed into the LaTeX system files,
% manually specify the path to it like:
% \documentclass[journal]{../sty/IEEEtran}

%% lel



% Some very useful LaTeX packages include:
% (uncomment the ones you want to load)


% *** MISC UTILITY PACKAGES ***
%
%\usepackage{ifpdf}
% Heiko Oberdiek's ifpdf.sty is very useful if you need conditional
% compilation based on whether the output is pdf or dvi.
% usage:
% \ifpdf
%   % pdf code
% \else
%   % dvi code
% \fi
% The latest version of ifpdf.sty can be obtained from:
% http://www.ctan.org/tex-archive/macros/latex/contrib/oberdiek/
% Also, note that IEEEtran.cls V1.7 and later provides a builtin
% \ifCLASSINFOpdf conditional that works the same way.
% When switching from latex to pdflatex and vice-versa, the compiler may
% have to be run twice to clear warning/error messages.






% *** CITATION PACKAGES ***
%
\usepackage{cite}
% cite.sty was written by Donald Arseneau
% V1.6 and later of IEEEtran pre-defines the format of the cite.sty package
% \cite{} output to follow that of IEEE. Loading the cite package will
% result in citation numbers being automatically sorted and properly
% "compressed/ranged". e.g., [1], [9], [2], [7], [5], [6] without using
% cite.sty will become [1], [2], [5]--[7], [9] using cite.sty. cite.sty's
% \cite will automatically add leading space, if needed. Use cite.sty's
% noadjust option (cite.sty V3.8 and later) if you want to turn this off.
% cite.sty is already installed on most LaTeX systems. Be sure and use
% version 4.0 (2003-05-27) and later if using hyperref.sty. cite.sty does
% not currently provide for hyperlinked citations.
% The latest version can be obtained at:
% http://www.ctan.org/tex-archive/macros/latex/contrib/cite/
% The documentation is contained in the cite.sty file itself.






% *** GRAPHICS RELATED PACKAGES ***
%
\ifCLASSINFOpdf
   \usepackage[pdftex]{graphicx}
  % declare the path(s) where your graphic files are
   \graphicspath{{../pdf/}{../jpeg/}}
  % and their extensions so you won't have to specify these with
  % every instance of \includegraphics
   \DeclareGraphicsExtensions{.pdf,.jpeg,.png}
\else
  % or other class option (dvipsone, dvipdf, if not using dvips). graphicx
  % will default to the driver specified in the system graphics.cfg if no
  % driver is specified.
   \usepackage[dvips]{graphicx}
  % declare the path(s) where your graphic files are
   \graphicspath{{../eps/}}
  % and their extensions so you won't have to specify these with
  % every instance of \includegraphics
   \DeclareGraphicsExtensions{.eps}
\fi
% graphicx was written by David Carlisle and Sebastian Rahtz. It is
% required if you want graphics, photos, etc. graphicx.sty is already
% installed on most LaTeX systems. The latest version and documentation can
% be obtained at: 
% http://www.ctan.org/tex-archive/macros/latex/required/graphics/
% Another good source of documentation is "Using Imported Graphics in
% LaTeX2e" by Keith Reckdahl which can be found as epslatex.ps or
% epslatex.pdf at: http://www.ctan.org/tex-archive/info/
%
% latex, and pdflatex in dvi mode, support graphics in encapsulated
% postscript (.eps) format. pdflatex in pdf mode supports graphics
% in .pdf, .jpeg, .png and .mps (metapost) formats. Users should ensure
% that all non-photo figures use a vector format (.eps, .pdf, .mps) and
% not a bitmapped formats (.jpeg, .png). IEEE frowns on bitmapped formats
% which can result in "jaggedy"/blurry rendering of lines and letters as
% well as large increases in file sizes.
%
% You can find documentation about the pdfTeX application at:
% http://www.tug.org/applications/pdftex





% *** MATH PACKAGES ***
%
\usepackage[cmex10]{amsmath}
% A popular package from the American Mathematical Society that provides
% many useful and powerful commands for dealing with mathematics. If using
% it, be sure to load this package with the cmex10 option to ensure that
% only type 1 fonts will utilized at all point sizes. Without this option,
% it is possible that some math symbols, particularly those within
% footnotes, will be rendered in bitmap form which will result in a
% document that can not be IEEE Xplore compliant!
%
% Also, note that the amsmath package sets \interdisplaylinepenalty to 10000
% thus preventing page breaks from occurring within multiline equations. Use:
%\interdisplaylinepenalty=2500
% after loading amsmath to restore such page breaks as IEEEtran.cls normally
% does. amsmath.sty is already installed on most LaTeX systems. The latest
% version and documentation can be obtained at:
% http://www.ctan.org/tex-archive/macros/latex/required/amslatex/math/





% *** SPECIALIZED LIST PACKAGES ***
%
\usepackage{algorithmic}
% algorithmic.sty was written by Peter Williams and Rogerio Brito.
% This package provides an algorithmic environment fo describing algorithms.
% You can use the algorithmic environment in-text or within a figure
% environment to provide for a floating algorithm. Do NOT use the algorithm
% floating environment provided by algorithm.sty (by the same authors) or
% algorithm2e.sty (by Christophe Fiorio) as IEEE does not use dedicated
% algorithm float types and packages that provide these will not provide
% correct IEEE style captions. The latest version and documentation of
% algorithmic.sty can be obtained at:
% http://www.ctan.org/tex-archive/macros/latex/contrib/algorithms/
% There is also a support site at:
% http://algorithms.berlios.de/index.html
% Also of interest may be the (relatively newer and more customizable)
% algorithmicx.sty package by Szasz Janos:
% http://www.ctan.org/tex-archive/macros/latex/contrib/algorithmicx/




% *** ALIGNMENT PACKAGES ***
%
\usepackage{array}
% Frank Mittelbach's and David Carlisle's array.sty patches and improves
% the standard LaTeX2e array and tabular environments to provide better
% appearance and additional user controls. As the default LaTeX2e table
% generation code is lacking to the point of almost being broken with
% respect to the quality of the end results, all users are strongly
% advised to use an enhanced (at the very least that provided by array.sty)
% set of table tools. array.sty is already installed on most systems. The
% latest version and documentation can be obtained at:
% http://www.ctan.org/tex-archive/macros/latex/required/tools/


\usepackage{mdwmath}
\usepackage{mdwtab}
% Also highly recommended is Mark Wooding's extremely powerful MDW tools,
% especially mdwmath.sty and mdwtab.sty which are used to format equations
% and tables, respectively. The MDWtools set is already installed on most
% LaTeX systems. The lastest version and documentation is available at:
% http://www.ctan.org/tex-archive/macros/latex/contrib/mdwtools/


% IEEEtran contains the IEEEeqnarray family of commands that can be used to
% generate multiline equations as well as matrices, tables, etc., of high
% quality.


%\usepackage{eqparbox}
% Also of notable interest is Scott Pakin's eqparbox package for creating
% (automatically sized) equal width boxes - aka "natural width parboxes".
% Available at:
% http://www.ctan.org/tex-archive/macros/latex/contrib/eqparbox/





% *** SUBFIGURE PACKAGES ***
\usepackage[tight,footnotesize]{subfigure}
% subfigure.sty was written by Steven Douglas Cochran. This package makes it
% easy to put subfigures in your figures. e.g., "Figure 1a and 1b". For IEEE
% work, it is a good idea to load it with the tight package option to reduce
% the amount of white space around the subfigures. subfigure.sty is already
% installed on most LaTeX systems. The latest version and documentation can
% be obtained at:
% http://www.ctan.org/tex-archive/obsolete/macros/latex/contrib/subfigure/
% subfigure.sty has been superceeded by subfig.sty.

\usepackage{todonotes}
\usepackage{multirow}
\newcommand{\inote}[1] {\todo[inline]{#1}}
\usepackage{longtable}



\usepackage{caption}
%\usepackage[font=footnotesize]{subfig}
% subfig.sty, also written by Steven Douglas Cochran, is the modern
% replacement for subfigure.sty. However, subfig.sty requires and
% automatically loads Axel Sommerfeldt's caption.sty which will override
% IEEEtran.cls handling of captions and this will result in nonIEEE style
% figure/table captions. To prevent this problem, be sure and preload
% caption.sty with its "caption=false" package option. This is will preserve
% IEEEtran.cls handing of captions. Version 1.3 (2005/06/28) and later 
% (recommended due to many improvements over 1.2) of subfig.sty supports
% the caption=false option directly:
%\usepackage[caption=false,font=footnotesize]{subfig}
%
% The latest version and documentation can be obtained at:
% http://www.ctan.org/tex-archive/macros/latex/contrib/subfig/
% The latest version and documentation of caption.sty can be obtained at:
% http://www.ctan.org/tex-archive/macros/latex/contrib/caption/

\usepackage[acronym]{glossaries}


% *** FLOAT PACKAGES ***
%
\usepackage{fixltx2e}
% fixltx2e, the successor to the earlier fix2col.sty, was written by
% Frank Mittelbach and David Carlisle. This package corrects a few problems
% in the LaTeX2e kernel, the most notable of which is that in current
% LaTeX2e releases, the ordering of single and double column floats is not
% guaranteed to be preserved. Thus, an unpatched LaTeX2e can allow a
% single column figure to be placed prior to an earlier double column
% figure. The latest version and documentation can be found at:
% http://www.ctan.org/tex-archive/macros/latex/base/



\usepackage{stfloats}
% stfloats.sty was written by Sigitas Tolusis. This package gives LaTeX2e
% the ability to do double column floats at the bottom of the page as well
% as the top. (e.g., "\begin{figure*}[!b]" is not normally possible in
% LaTeX2e). It also provides a command:
%\fnbelowfloat
% to enable the placement of footnotes below bottom floats (the standard
% LaTeX2e kernel puts them above bottom floats). This is an invasive package
% which rewrites many portions of the LaTeX2e float routines. It may not work
% with other packages that modify the LaTeX2e float routines. The latest
% version and documentation can be obtained at:
% http://www.ctan.org/tex-archive/macros/latex/contrib/sttools/
% Documentation is contained in the stfloats.sty comments as well as in the
% presfull.pdf file. Do not use the stfloats baselinefloat ability as IEEE
% does not allow \baselineskip to stretch. Authors submitting work to the
% IEEE should note that IEEE rarely uses double column equations and
% that authors should try to avoid such use. Do not be tempted to use the
% cuted.sty or midfloat.sty packages (also by Sigitas Tolusis) as IEEE does
% not format its papers in such ways.


%\ifCLASSOPTIONcaptionsoff
%  \usepackage[nomarkers]{endfloat}
% \let\MYoriglatexcaption\caption
% \renewcommand{\caption}[2][\relax]{\MYoriglatexcaption[#2]{#2}}
%\fi
% endfloat.sty was written by James Darrell McCauley and Jeff Goldberg.
% This package may be useful when used in conjunction with IEEEtran.cls'
% captionsoff option. Some IEEE journals/societies require that submissions
% have lists of figures/tables at the end of the paper and that
% figures/tables without any captions are placed on a page by themselves at
% the end of the document. If needed, the draftcls IEEEtran class option or
% \CLASSINPUTbaselinestretch interface can be used to increase the line
% spacing as well. Be sure and use the nomarkers option of endfloat to
% prevent endfloat from "marking" where the figures would have been placed
% in the text. The two hack lines of code above are a slight modification of
% that suggested by in the endfloat docs (section 8.3.1) to ensure that
% the full captions always appear in the list of figures/tables - even if
% the user used the short optional argument of \caption[]{}.
% IEEE papers do not typically make use of \caption[]'s optional argument,
% so this should not be an issue. A similar trick can be used to disable
% captions of packages such as subfig.sty that lack options to turn off
% the subcaptions:
% For subfig.sty:
% \let\MYorigsubfloat\subfloat
% \renewcommand{\subfloat}[2][\relax]{\MYorigsubfloat[]{#2}}
% For subfigure.sty:
% \let\MYorigsubfigure\subfigure
% \renewcommand{\subfigure}[2][\relax]{\MYorigsubfigure[]{#2}}
% However, the above trick will not work if both optional arguments of
% the \subfloat/subfig command are used. Furthermore, there needs to be a
% description of each subfigure *somewhere* and endfloat does not add
% subfigure captions to its list of figures. Thus, the best approach is to
% avoid the use of subfigure captions (many IEEE journals avoid them anyway)
% and instead reference/explain all the subfigures within the main caption.
% The latest version of endfloat.sty and its documentation can obtained at:
% http://www.ctan.org/tex-archive/macros/latex/contrib/endfloat/
%
% The IEEEtran \ifCLASSOPTIONcaptionsoff conditional can also be used
% later in the document, say, to conditionally put the References on a 
% page by themselves.





% *** PDF, URL AND HYPERLINK PACKAGES ***
%
\usepackage{url}
%\usepackage{hyperref}
% url.sty was written by Donald Arseneau. It provides better support for
% handling and breaking URLs. url.sty is already installed on most LaTeX
% systems. The latest version can be obtained at:
% http://www.ctan.org/tex-archive/macros/latex/contrib/misc/
% Read the url.sty source comments for usage information. Basically,
% \url{my_url_here}.

\newcommand\MYhyperrefoptions{bookmarks=true,bookmarksnumbered=true,
pdfpagemode={UseOutlines},plainpages=false,pdfpagelabels=true,
colorlinks=true,linkcolor={black},citecolor={black},urlcolor={black},
pdftitle={Emerging Nuclear Reactor Technology for Near-Mid Term Commercial Deployment},%<!CHANGE!
pdfsubject={Nuclear Reactor Technology},%<!CHANGE!
pdfauthor={Thomas J. Smith},%<!CHANGE!
pdfkeywords={Nuclear, Technology, Emerging, Commercial, Gen IV, Gen III+, Gen III}}%<^!CHANGE!
\ifCLASSINFOpdf
\usepackage[\MYhyperrefoptions,pdftex]{hyperref}
\else
\usepackage[\MYhyperrefoptions,breaklinks=true,dvips]{hyperref}
\usepackage{breakurl}
\fi
% *** Do not adjust lengths that control margins, column widths, etc. ***
% *** Do not use packages that alter fonts (such as pslatex).         ***
% There should be no need to do such things with IEEEtran.cls V1.6 and later.
% (Unless specifically asked to do so by the journal or conference you plan
% to submit to, of course. )


% correct bad hyphenation here
\hyphenation{op-tical net-works semi-conduc-tor}


\begin{document}
%
% paper title
% can use linebreaks \\ within to get better formatting as desired
\title{Emerging Nuclear Reactor Technology for Mid Term Commercial Deployment}
%
%
% author names and IEEE memberships
% note positions of commas and nonbreaking spaces ( ~ ) LaTeX will not break
% a structure at a ~ so this keeps an author's name from being broken across
% two lines.
% use \thanks{} to gain access to the first footnote area
% a separate \thanks must be used for each paragraph as LaTeX2e's \thanks
% was not built to handle multiple paragraphs
%

\author{\textbf{\href{mailto:tjs1g10@ecs.soton.ac.uk}{Thomas~J.~Smith}},\\ \emph{\small{University of Southampton, Faculty of Physical and Applied Sciences, Electronics and Computer Science}}}% <-this % stops a space
%\thanks{M. Shell is with the Department
%of Electrical and Computer Engineering, Georgia Institute of Technology, Atlanta,
%GA, 30332 USA e-mail: (see http://www.michaelshell.org/contact.html).}% <-this % stops a space
%\thanks{J. Doe and J. Doe are with Anonymous University.}% <-this % stops a space
%\thanks{Manuscript received April 19, 2005; revised January 11, 2007.}}

% note the % following the last \IEEEmembership and also \thanks - 
% these prevent an unwanted space from occurring between the last author name
% and the end of the author line. i.e., if you had this:
% 
% \author{....lastname \thanks{...} \thanks{...} }
%                     ^------------^------------^----Do not want these spaces!
%
% a space would be appended to the last name and could cause every name on that
% line to be shifted left slightly. This is one of those "LaTeX things". For
% instance, "\textbf{A} \textbf{B}" will typeset as "A B" not "AB". To get
% "AB" then you have to do: "\textbf{A}\textbf{B}"
% \thanks is no different in this regard, so shield the last } of each \thanks
% that ends a line with a % and do not let a space in before the next \thanks.
% Spaces after \IEEEmembership other than the last one are OK (and needed) as
% you are supposed to have spaces between the names. For what it is worth,
% this is a minor point as most people would not even notice if the said evil
% space somehow managed to creep in.



% The paper headers
\markboth{COMP6033 Independent Research Review}%
{COMP6033 Independent Research Review}
% The only time the second header will appear is for the odd numbered pages
% after the title page when using the twoside option.
% 
% *** Note that you probably will NOT want to include the author's ***
% *** name in the headers of peer review papers.                   ***
% You can use \ifCLASSOPTIONpeerreview for conditional compilation here if
% you desire.




% If you want to put a publisher's ID mark on the page you can do it like
% this:
%\IEEEpubid{0000--0000/00\$00.00~\copyright~2007 IEEE}
% Remember, if you use this you must call \IEEEpubidadjcol in the second
% column for its text to clear the IEEEpubid mark.



% use for special paper notices
%\IEEEspecialpapernotice{(Invited Paper)}




% make the title area
\maketitle


\begin{abstract}
\boldmath
The recently renewed global interest has restarted research into next generation nuclear reactor technology. This paper looks at the current state of technology worldwide and attempts to establish some of the promising areas of future reactor research. Looking at these technologies in the light of commercial viability presents an interesting new outlook on nuclear technology. Or something along these lines\dots
\end{abstract}
% IEEEtran.cls defaults to using nonbold math in the Abstract.
% This preserves the distinction between vectors and scalars. However,
% if the journal you are submitting to favors bold math in the abstract,
% then you can use LaTeX's standard command \boldmath at the very start
% of the abstract to achieve this. Many IEEE journals frown on math
% in the abstract anyway.

% Note that keywords are not normally used for peerreview papers.
%\begin{IEEEkeywords}
%IEEEtran, journal, \LaTeX, paper, template.
%\end{IEEEkeywords}






% For peer review papers, you can put extra information on the cover
% page as needed:
% \ifCLASSOPTIONpeerreview
% \begin{center} \bfseries EDICS Category: 3-BBND \end{center}
% \fi
%
% For peerreview papers, this IEEEtran command inserts a page break and
% creates the second title. It will be ignored for other modes.
%\IEEEpeerreviewmaketitle

\section{Introduction}

\IEEEPARstart{E}{nergy} security, sustainability and affordability are key challenges confronting global energy policy. 
2050 electricity demand is expected to increase by a minimum of 27\% compared to 2010. 
This is attributed to the electrification of transport and heating in developed countries and increasing demand in developing countries \cite{Kim201}.
This insatiable demand for more energy must be reconciled with combating climate change and lowering greenhouse gas emissions, and has led to a rekindled interest in nuclear power in recent years \cite{schneider2012nuclear}.

Generating electricity from nuclear fission is a well established science and has been used to supply low-carbon power in commercial quantities since Calder Hall opened in October 1956 \cite{NDA2007}.
Most of the worlds' reactors were built between 1965 and 1975 with the peak number of grid connections in 1984 \cite{Guidolin20121746,schneider2012nuclear}, just prior to the Chernobyl catastrophe in 1986.
By the end of the 1980s, the upward trend in nuclear capacity had ended \cite{schneider2012nuclear}.

While a full scale ``nuclear renaissance'' was thwarted by the multi-reactor incident at Fukushima in 2011 \cite{Guidolin20121746}, nuclear energy remains an important base-load resource in the low-carbon energy mix. 
%In 2013, the IAEA recorded 437 operational reactors and a further 67 reactors under construction \cite{IAEAReactors2013}.
In 2012, 437 nuclear fission reactors generated 10.3\% of the worlds' electricity\footnote{Calculated from IAEA nuclear total generation \cite{IAEAReactors2013} of 2,346TWh and Enerdata total electricity production value of 22,619TWh \cite{Enerdata}.}. 
A further 67 reactors were under construction \cite{IAEAReactors2013}, more than at any point since 1988 \cite{schneider2012nuclear}.
As a new generation of reactors enter the construction phase, it is pertinent to consider the future design and challenges faced in new revolutionary designs.

\section{Evolution of Reactor Technology}
Nuclear energy systems are generally split into four generations of technology. 
The first experimental and prototype systems were built between 1950 and 1970 \cite{GenIVRoadmap} and are referred to as Generation I.
These include examples of PWR, BWR, Magnox and fast reactor technologies \cite{goldberg2011nuclear}\cite{Bhatnagar2011}. 

Generation II systems are reactors that were deployed between 1970 and 1990, following the OPEC-oil crisis in 1974 \cite{Bhatnagar2011}.
These commercial sized systems form the bulk of the reactors operating today. The majority of Gen II systems use PWR technology, although there is a significant numbers of BWRs, and a minority of AGRs, VVER and CANDU reactors \cite{IAEAReactors2013}. 
These reactors are claimed to have exemplary safety and commercial performance \cite{Locatelli2013, goldberg2011nuclear, Bhatnagar2011, GenIVRoadmap} despite the incidents at Chernobyl, Three Mile Island and Fukushima.

Many of the 67 reactors currently under construction will be Gen III or III+ designs \cite{IAEAReactors2013}, representing evolutionary improvements over Gen II reactor designs \cite{Marques2010a}. 
Such improvements include the standardisation of designs for licensing purposes, simpler modular designs, the greater use of passive safety systems and reduced core melt frequency \cite{schneider2012nuclear, Marques2010a}. 
Gen III systems are currently being considered for, or under construction at \cite{IAEAReactors2013}:
\begin{itemize}
\item Areva EPR - Flamanville France, Olkiluoto Finland, Hinkley Point and Sizewell UK
\item Westinghouse AP-1000 - Moorside UK, Sanmen and Haiyang China, Vogtle USA
\item GE Hitachi ES-BWR - Fermi USA
\end{itemize}
These systems are expected to continue to dominate the new build market for the coming decade \cite{Marques2010a}.

This paper is concerned with the future of reactor technology. Gen IV technology has four goals; to improve the sustainability, economics, safety performance and proliferation resistance of nuclear energy systems \cite{GenIVRoadmap}. 
These designs are likely to be available for commercial deployment beyond 2030, when most of the world's current reactors will reach the end of their operational life \cite{schneider2012nuclear}. 
The Generation IV International Forum (GIF) was established in 2001 to coordinate the research required to realise next generation reactor designs \cite{Marques2010a, GenIVRoadmap}. 
Six key emerging technologies are highlighted by the Gen IV International Forum \cite{GenIVForum} and are introduced in Table \ref{table:GenIV}.

\begin{table}[!t]
% increase table row spacing, adjust to taste
\renewcommand{\arraystretch}{1.3}
% if using array.sty, it might be a good idea to tweak the value of
% \extrarowheight as needed to properly center the text within the cells
\caption{Gen IV Technology Summary \cite{GenIVRoadmap, Marques2010a}}
\label{table:GenIV}
\centering
% Some packages, such as MDW tools, offer better commands for making tables
% than the plain LaTeX2e tabular which is used here.
\begin{tabular}{|c|c|c|}
\hline
\textbf{Reactor System} & \textbf{Spectrum} & \textbf{Fuel Cycle}\\
\hline
Sodium Cooled Fast Reactor (SFR) & Fast & Closed \\
\hline
Very High Temperature Reactor (VHTR) & Thermal & Open \\
\hline
Supercritical Water Cooled Reactor (SCWR) & Thermal \& Fast & Open \& Closed \\
\hline
Gas Cooled Fast Reactor (GFR) & Fast & Closed \\
\hline
Lead Cooled Fast Reactor (LFR) & Fast & Closed \\
\hline
Molten Salt Reactor (MSR) & Thermal & Closed \\
\hline

\end{tabular}
\end{table}

The underpinning concepts of the highlighted Gen IV technologies are not entirely new. 
All of these technologies have been previously studied in various levels of detail prior to the formation of the GIF \cite{Marques2010a}. However, there is a marked removal from past approaches. 
Three of the six Gen IV technologies are fast reactors as opposed to the thermal reactor designs dominating past generations.
Fast reactors utilise unmoderated neutrons that have the capacity to convert fertile material into fissile fuel, increasing uranium efficiency by around 50 times \cite{Locatelli2013}. 
Four of the six technologies use a closed fuel cycle as opposed to the once-through cycle currently used. 
Closing the fuel cycle presents significant societal benefits, namely the reduction of nuclear waste through better actinide management \cite{GenIVRoadmap, Bhatnagar2011}. 
Gen IV systems will achieve much  higher operating temperatures than current designs, promoting the use of nuclear energy for co-generation. 
Hydrogen produced in large quantities by high temperature reactors could be a critical energy vector in the future \cite{ Bhatnagar2011}. 

In the following sections, the state-of-the-art research into the highlighted technologies is collated and presented.


\section{Sodium Cooled Fast Reactor}
SFRs are one of the most mature Gen IV technologies \cite{Marques2010a}. 
Worldwide investment in the development and demonstration of SFRs already exceeds US \$50billion \cite{Int2012}. 
By utilising fast-spectrum neutrons and closed fuel recycling, SFRs present significant improvements in fuel utilisation and is one of the nearer term options for managing plutonium and minor actinide waste \cite{GenIVForum}.

Liquid sodium is an excellent coolant. 
It is inexpensive, liquid within a wide operating range $93^{o}C-883^{o}C$ and undergoes very little activation \cite{Sakamoto2013194}. 
Sodium is chemically compatible with stainless steel when closed within the reactor, but reacts violently with air and water \cite{Int2012, Sakamoto2013194} which has caused several fires at SFR experimental facilities.
Although none of these fires have caused any radiological leakage \cite{Int2012}, there is a continuous effort to reduce leak occurrence and mitigate the fire risk. 
Various techniques have been developed including the Karlsruhe tray \cite{Huber1975155}, implemented at Superph\'{e}nix in France \cite{Int2012}, whereby leaking sodium is caught and stored in an air and water free environment. 
The Gen IV forum are actively addressing safety concerns such as these \cite{GenIVRoadmap}.

There are several potential configurations of SFR concerning power output and layout. 
The layout could be chosen to be a pool-design as experimentally proven at EBR-II in the US, or a loop design as first proven at BR-10 in Russia \cite{Int2012}. 
The JSFR is a 3570MWth SFR design consideration in Japan and will use a loop configuration. 
Prior to construction starting, the two layouts were compared considering criteria such as construction cost, technical feasibility, In Service Inspection \& Repair (ISIR) and potential for cost reduction \cite{Sakamoto2013194}.
The pool layout presented some significant advantages including the reduction in leakage probability due to a fully enclosed primary circuit, higher thermal inertia which is beneficial in an accident scenario and has is the most experimented design.
%\begin{itemize}
%\item reduction in leakage probability due to fully enclosed primary circuit.
%\item higher thermal inertia which is beneficial in an accident scenario.
%\item most experimented form.
%\end{itemize}
but the loop layout was chosen due to economic competitiveness \cite{Locatelli2013}. 
The loop design has scope for cost savings through changes to the secondary loop, including the reduction in piping length and the integration of the intermediate heat exchanger and primary circulation pump \cite{Sakamoto2013194}. 
Other SFR designs such as Korean KALIMER design and the BN-800 currently under construction in Russia and China adopt a pool design. Despite the Japanese claims, the pool layout also presents opportunity for cost efficiency by reducing piping length and adopting innovative core internals \cite{Locatelli2013}. It remains to be seen which of these designs prevail, and will largely come down to the combined construction and operational costs which will be verified by these demonstration projects.

Another key design decision is the size of the reactor. 
While it is necessary to consider the economies of scale, not all SFR designs are large commercial power reactors greater than 1000MWth.
There are several Small Modular Reactor (SMR) designs utilising sodium as coolant, including the Toshiba 4S, a pool design of just 10MWe \cite{Locatelli2013}. 
The PRISM reactor offered by GE Hitachi is another pool type SFR of 311MWe output \cite{Locatelli2013}. 
The NDA is considering deploying a PRISM reactor in the UK to deal with the 140 tonne stockpile of civil plutonium \cite{NDA14}. 
The design is based on the EBR-II reactor in the USA, and this experience-based design combined with a small unit size means the licensing and construction time combined is just 14-18 years\cite{NDA14}.

There remains several performance related challenges to overcome before the SFR can be deployed commercially. 
SFRs have been built and operated successfully in France, Germany, India, Japan, Russia, the UK and the USA meaning a considerable amount of construction and operating experience has already been gained \cite{Int2012}.
The GIF has organised these challenges into several working groups \cite{GenIVForum}:
\begin{itemize}
\item System Integration and Assessment Project - to track and integrate global R\&D approaches and experience.
\item Safety and Operations Project - to continue model development, experimental programs and assessment of safety systems. Also to integrate decommissioning feedback, ISIR techniques and under-sodium viewing techniques.
\item Advanced Fuel Project - to design high burn-up minor actinide bearing fuels, as well as associated cladding and wrappers.
\item Component Design and Balance-of-Plant Project - to enhance the performance of the SFR for economic feasibility. 
\item Gloabal Actinide Cycle International Demonstration Project - to demonstrate on a significant scale actinide management by fast reactors.
\end{itemize}


\section{Very High Temperature Reactor}
The VHTR concept is a helium gas cooled, graphite moderated design planned for an open fuel cycle \cite{Marques2010a}.
The main purpose of this reactor is to achieve extremely high core outlet temperatures. The Gen IV forum target for this technology is to reach over $1000^oC$ \cite{GenIVRoadmap} as opposed to $320^oC$ for current LWRs and around $500^oC-700^oC$ for Gen IV SFRs and MSRs \cite{Bhatnagar2011}.
The increased temperature improves the thermodynamic efficiency of the system.
The GT-MHR developed by General Atomics is expected to reach efficiencies of 47\% operating at $850^oC$ \cite{Marques2010a} compared to around 30\% for current LWR designs.
More importantly, the high outlet temperature increases the potential for co-generation.
Process heat is currently used in oil refining, metallurgic processes and synthetic hydrocarbon production.
Temperatures in excess of $750^oC$ are required for hydrogen production, either by the thermo-chemical iodine/sulphur process or the high-temperature electrolysis of steam.
Hydrogen, like electricity or refined hydrocarbons, is an energy vector and potentially significant in a post-petroleum society.
The VHTR is one of the few clean technologies that could be used to create abundant quantities of hydrogen \cite{Bhatnagar2011}.

The VHTR could use either a pebble bed core or a prismatic block core.
The HTTR in Japan started operating in 2004 with a prismatic core, while the Chinese HTR-10 uses a pebble bed core \cite{Ftterer2014}.
There are some trade-offs between the designs.
The pebble bed core has a lower operating temperature allowing the use of a conventional steal pressure vessel and has an increased capacity factor due to online refuelling.
However, the prismatic core has a higher power density making it economically superior \cite{Locatelli2013}.
The technology readiness level of both core designs is relatively high due to these experimental projects \cite{Ftterer2014}.
Regardless of the core design, a significantly different fuel design is used.
Tri-isotropic (TRISO) layered particles are 1mm diameter spheres containing a fuel kernel at the centre.
The kernel is covered in three layers of carbon and a layer of silicon carbide.
This fuel design is required to withstand the high temperatures, and also improves proliferation resistance of the system by enclosing all fuel inside the particle \cite{Marques2010a}.

There remains some significant challenges to the commercial deployment of the VHTR. 
The GIF coordinate the global research programme and have four paths of research \cite{GenIVForum}.
The identification of high temperature materials for pipework, gas circulators, heat exchangers and other core internals is of critical importance. 
Data is currently being gathered on the performance of several alloys at very high temperature for collation by ASME, expected to be complete in the next four years \cite{Ftterer2014}.
The TRISO fuel is also a subject of a GIF working group, which aims to establish the best practice for TRISO fuel fabrication, undertake irradiation experiments \cite{Ftterer2014} and to understand how to reprocess spent fuel \cite{GenIVForum}. 
The hydrogen production group examine both thermo-chemical and high-temperature electrolysis hydrogen production methods \cite{GenIVForum}.
Despite this being a longer term goal, the future potential of a large scale hydrogen economy warrants the research effort.
Current work focusses on producing a $50-200l/h$ laboratory scale hydrogen production example with the thermo-chemical method \cite{Ftterer2014}.
The final working group involves computational modelling for verification and benchmarking for licensing and design improvement \cite{GenIVForum}.

The HTR-PM is a commercial evolution of the HTR-10 reactor up-scaled to 210MWe using a pebble bed core and is being built in Shidaowan, China \cite{Locatelli2013}.
Although commercial deployment of VHTRs is not expected until 2020 \cite{GenIVRoadmap}, construction of the HTR-PM is expected to be complete in 2017 \cite{Ftterer2014}. 
The VHTR design presents some significant design challenges, but as one of the more realistic options for large scale hydrogen production, this design could be deployed widely in the future.
The 2025 target deployment date set by the GIF \cite{GenIVRoadmap} is attainable provided the development of commercial partnerships to capitalise on the current research \cite{Ftterer2014}.
%% 02/03/14 Spell Check to Here

\section{Gas Cooled Fast Reactor}
Following on from the VHTR, the GFR is another gas cooled, high temperature design.
A helium coolant is used but no moderator is required due to the use of the fast neutron spectrum.
This allows the use of a closed fuel cycle for full actinide management and the use of both fissile and fertile fuel.
This is expected to deliver fuel efficiencies two orders of magnitude greater than current LWR designs \cite{GenIVRoadmap}.
It will operate at similar temperatures to the VHTR of around $850^oC$, which provides opportunities for cogeneration \cite{Bhatnagar2011}.

The reference GFR design is 2400MWth in an attempt to exploit economies of scale \cite{Locatelli2013}.
The system uses a helium coolant in a primary circuit, which is used to heat a secondary helium-nitrogen mixture through a heat exchanger.
This secondary circuit gas operates on a closed helium Brayton cycle.
Waste heat from the gas turbine is recovered through a conventional steam Rankine cycle \cite{GenIVForum}.

The use of helium as a coolant has some significant advantages.
It is single phase, chemically inert, non-corrosive, transparent to neutrons and optically transparent \cite{Int2012}.
This means that directly coupled Brayton cycles can be used without specialist equipment to isolate the coolant or for fire protection and also vastly simplifies ISIR techniques through simple observation \cite{Locatelli2013}.
However, helium does not have the same heat transfer properties as sodium or metal coolants and must be pressurised.
Additionally, the GFR has no graphite moderator which enhances the thermal inertia of the core under and accident scenario in the VHTR.
There is some significant R\&D required to identify passive cooling and pressurisation techniques for accident scenarios \cite{Int2012}.

The GFR and VHTR have some synergistic research areas, particularly high temperature materials, the development of helium turbomachinery and the development of advanced nuclear fuel \cite{Locatelli2013}. 
The 75MWth ALLEGRO experimental reactor is a key enabler of GFR research and critical to achieve demonstration of many key aspects \cite{Int2012} and planned for commissioning in 2026 \cite{Locatelli2013}.
The aim of ALLEGRO is to demonstrate commercial viability of GFRs, demonstrate safe core operation with appropriate instrumentation, establish an operational safety framework and to gain operating experience \cite{Int2012}.
Current research for ALLEGRO focuses on fuel design and reactor modelling for safety system development \cite{GenIVForum}.

The target commercial deployment date for the GFR was 2025 when the GIF produced the 2002 Gen IV roadmap \cite{GenIVRoadmap}. 
However, with ALLEGRO not planned for commissioning until 2026, it is highly unlikely this target will be met.
France and Japan have taken a lead role in developing the GFR.
Since Fukushima France has replaced a focus on the more developed SFR concept and Japan has also slowed research \cite{GenIVForum}.
The GFR is more likely to be deployed as an evolution of the VHTR thermal reactor design further into the future.

\section{Supercritical Water Cooled Reactor}
The main design aim of the SCWR is to increase the efficiency of current LWR reactor designs by operating at high temperatures and pressures, above the thermodynamic critical point of water \cite{GenIVRoadmap}.
The SCWR concept has been inspired by state-of-the-art coal-fired power stations that have successfully increased net efficiency while decreasing the capital cost required per unit electricity.
Supercritical coal plants are expected to reach efficiencies of 50\% in the near future while current LWR designs are only around 30\% efficient and have not improved since the 1970s \cite{GenIVForum}.
The target efficiency for the SCWR is 44\% \cite{Marques2010a}.

Operating at supercritical pressures presents opportunities for large cost savings and safety improvements. 
Water can only exist in a single homogeneous phase above the thermodynamic critical point.
Current BWR designs have a number of in-core devices to handle the phase change of water as it circulates \cite{Marques2010a}.
SCWRs do not require devices such as steam dryers, separators, recirculation pumps or steam generators, which vastly simplifies the design.
This results in estimated savings of 50\% in specific capital cost and 35\% of O\&M costs when compared to current LWR designs \cite{GenIVRoadmap}.
This also has a positive impact on the safety performance of the reactor.
This is due to the simpler design alongside the elimination of core dry out events by the continuous heat transfer regime provided by a single phase coolant \cite{GenIVRoadmap}.

The technology base for SCWRs is broad. Supercritical coal-fired stations have operational experience involving turbogenerators, piping and other balance-of-plant equipment \cite{GenIVRoadmap}.
The reactor design could be either a pressure vessel similar to Gen III ABWRs, or a pressure tube design similar to the CANDU reactor \cite{Bhatnagar2011}.
The pressure vessel concept is being progressed in the Japanese Super LWR and European HPLWR projects. 
These are thermal designs optimised for electricity production expected for deployment is 2020 and 2035 respectively \cite{Locatelli2013}.
Additionally there is a variant on the Super LWR being developed for the fast neutron spectrum \cite{Locatelli2013}.
The pressure tube design is being developed in the Canadian-SCWR project, and could use a closed Thorium fuel cycle \cite{Locatelli2013}.

Unlike other Gen IV designs, the SCWR design has never been built.
This significantly delays the deployment of any SCWR designs for demonstration purposes \cite{GenIVForum}.
One of the main areas of research is into materials for both in and outside the reactor core.
Extreme materials are required to with stand the high temperatures and pressures, radiation and the highly corrosive environment due to the supercritical water \cite{GenIVForum}.
The next milestone for SCWR research is to conduct in-pile tests of a fuel assembly under supercritical conditions \cite{GenIVForum} and the Gen IV target for deployment of 2025 remains ambitious but attainable.

\section{Lead Cooled Fast Reactor}
The LFR refers to a fast reactor design cooled by one of two different heavy liquid metals. 
Lead (Pb) itself is one option and a lead-bismuth eutectic (LBE) composed of 45\% lead and 55\% bismuth the alternative \cite{Int2012}.
Both coolants are high temperature, low-pressure, chemically inert and feature good thermodynamic properties \cite{GenIVForum}.
Using a fast reactor with a closed fuel cycle means the LFR can be used as a burner to consume minor actinide waste or as a breeder using thorium matrices \cite{GenIVForum}.
Some believe that LFR designs can increase the safety of reactor systems while being economically competitive with LWRs \cite{Int2012}.

A pure Pb coolant is the preferred option since PB is inexpensive and abundant, while Bismuth is a rare earth element.
Pb is also less corrosive at high temperatures than a LBE \cite{Locatelli2013}.
However, the LBE has a melting point of $125^{o}C$ compared to pure Pb at $327^{o}C$ and therefore reduces the risk of core freezing during transient events \cite{Locatelli2013}.
Techniques for maintenance at high temperature are required in order to use a pure Pb coolant \cite{Int2012}.
A further advantage of a pure Pb coolant is that Bismuth undergoes activation in high neutron flux.
Using LBE produces amounts of $^{210}$Po which requires a complex and expensive srubbing process for removal \cite{Int2012}.

The LFR has further advantages besides those shared with all fast reactor designs.
Pb and LBE do not react with air or water meaning a secondary circuit is not a necessary safety requirement.
Many designs feature the steam generators immersed in the reactor pool core \cite{GenIVRoadmap}.
This reduces the capital cost of the plant and increases reliability when compared with SFRs for example.
Pb also has a high atmospheric boiling temperature of $1740^{o}C$ eliminating core boiling events \cite{Sakamoto2013194}.
It also permits high temperature operation to the limit of core materials, between $550^{o}C - 850^{o}C$, which increases the efficiency of attached thermodynamic cycles \cite{Marques2010a} .

There are several reference designs for the LFR concept, but this paper chooses two of significance.
The ELSY design is a 600MWe design with the aim to deploy a competitive fast reactor in the EU energy market \cite{Locatelli2013}.
It is a pool type design, 40\% efficient and operates at a relatively low maximum temperature of $550^{o}C$ to cater for current fuel cladding material technology \cite{Int2012}.
The project started in 2006 as a consortium of 20 organisations, however there is no expected deployment date for the system \cite{GenIVForum}.
The SSTAR project is a 20MWe SMR designed for easy transportation \cite{GenIVForum}.
The core is designed to last for 15-30 years without refueling and relies on natural convection for both operational and emergency heat removal scenarios \cite{Int2012}.
SSTAR could serve developing nations or remote communities with an integrally safe and reliable electricity supply while offering very high proliferation resistance \cite{Int2012}.

There remains some significant research required to meet the GIF target deployment date of 2025 \cite{GenIVRoadmap}.
The high melting point of Pb requires the system to be maintained at a high enough temperature to prevent solidification. 
This presents a challenge during repair and maintenance \cite{GenIVForum}.
Pb is opaque, a property shared with sodium, and requires advanced ISIR techniques to be developed \cite{GenIVForum}.
Lead coolants become increasingly corrosive to structural materials at high temperatures and flow rates.
Potential solutions include limiting flow rates below $1m/s$ or forming layers of protective oxide \cite{Int2012}.
Special materials must be developed for high velocity machinery parts, such as the internals of pumps.
Materials such as Ti$_3$SiC$_2$ are currently being tested for corrosion resistance for these purposes \cite{Int2012}.
A final challenge is presented by the natural toxicity of lead.
Regulations for construction workers have been developed which do not impede start-up or operation of experimental facilities \cite{Int2012}, but the disposal of such a coolant remains an ecological challenge that must be resolved \cite{Locatelli2013}.



\section{Molten Salt Reactor}
The uniquity of the MSR concept is that the nuclear fuel is dissolved in a molten mixture of fluoride salts \cite{GenIVRoadmap}.
Instead of holding the fuel in a solid phase encased in cladding, uranium and plutonium fluorides are dissolved in sodium or zirconium fluorides \cite{Marques2010a}.
The first MSR was experimentally operated at Oak Ridge in the USA in the 60s \cite{Marques2010a}.
This operated on the thermal spectrum with a graphite moderator.
Most research since 2005 has focussed on developing a fast spectrum MSR to combine the benefits of fast reactors and molten salt coolants and fuels \cite{GenIVForum}.

%Advantages
The advantages of this innovative fuel design is the elimination of fuel fabrication.
This vastly reduces fuel cycle costs \cite{Locatelli2013}.
The liquid fuel is spread throughout the core which eliminates any formation of hot spots associated with solid fuels \cite{Locatelli2013}, and also allows the adoption of a number of fuel cycles from burning LWR waste to breeding fissile $^{233}$U from thorium \cite{GenIVForum}.
Additionally, the coolant is optically transparent allowing traditional ISIR techniques to be employed \cite{Bhatnagar2011}.
Molten salts are thermally stable to a very high temperature, so the MSR can be operated up to $800^{o}C$ which promotes cogeneration applications \cite{Marques2010a}. 

%MOSART, MSFR and FHR.
There are several worldwide projects looking to implement MSR technology.
The French MSFR project is a fast reactor design exploiting the $^{232}$Th-$^{233}$U breeder fuel cycle \cite{GenIVForum} to produce 1300MWe \cite{Locatelli2013}.
The full design is yet to be finalised with multiple thermodynamic cycles for the secondary circuit considered, including the helium Brayton cycle and the supercritical Rankine cycle \cite{Locatelli2013}.
The project aims to build a prototype by 2020 and a commercial reactor by 2040 \cite{Locatelli2013}.
The Russian MOSART project is another fast reactor design but with a focus of actinide waste management. 
The key advantages of this design aside from waste management, is the ability to operate for up to 50 years without refueling by using a mixture of thorium and actinide waste fuels \cite{GenIVForum}.

%Generic Issues.
The use of a liquid fuel alongside a liquid coolant poses some issues that are specific to this concept \cite{Sakamoto2013194}.
The common aim of current GIF R\&D is to propose a single optimal system configuration from physical, chemical and material studies.
Initial physical studies indicate MSR designs have a very high negative temperature coefficients, which is beneficial in accident scenarios and absent in many Gen IV designs.
Safety considerations are paramount, and the negative temperature coefficient combined with the high thermal inertia of molten salts led to safe shutdown in major accident simulations \cite{GenIVForum}.
However, the passive safety of the design is yet to be established \cite{Locatelli2013}.
The salt coolant also poses other issues, including the development of a reprocessing strategy for a coolant/fuel mix and the understanding of the thermodynamic properties of molten salts \cite{GenIVForum}.
Material technology must also progress for the deployment of MSRs. Materials must be found that can withstand the corrosive, high temperature and high neutron flux conditions in the MSR core \cite{GenIVForum}.

The MSR poses some considerable design challenges.
The GIF target for commercial deployment of the MSR is 2025 \cite{GenIVRoadmap} which is very optimistic given the earliest planned project is 2040 \cite{Locatelli2013}.


\section{Challenges to Deployment}
%A summary of some of the broader challenges to deployment aside from the R\&D in each of the sections.
%Consideration of Fukushima, what if there was another Fukushima?
%Consideration of disposal of waste - actinide management of current generation?
%Consideration of political issues - eg Germany and Japan, social issues - perception of nuclear? Understanding of the challenges? Strike Price Agreements? and economic issues - financing first-of-a-kind reactor designs in a commercial power market? Is that possible?
%
%Need to consider how to condense this down and make it as coherent as possible.
%Expected to end page 5/start of page 6.
\section{Conclusion}
%Quick summary of points at the end.
%Is it likely that we will see any of these reactors deployed?
%If so which and why?
%Expected to halfway page 5.
%Leaves space for references and appendix.

\bibliographystyle{IEEEtran}
\bibliography{IRRBib}

\appendices
\section{List of Terms}
\footnotesize
\scriptsize
%\tiny
\textbf{4S} Super Safe, Small and Simple \par
\textbf{ABWR} Advanced Boiling Water Reactor\par
\textbf{ASME} American Society of Mechanical Engineers\par
\textbf{BWR} Boiling Water Reactor\par
\textbf{CANDU} CANadian Deuterium Uranium\par
\textbf{EBR} Experimental Breeder Reactor\par
\textbf{ELSY} European Lead-cooled SYstem\par
\textbf{EPR} European Pressurised Reactor\par
\textbf{ES-BWR} Economic Simplified Boiling Water Reactor\par
\textbf{GE} General Electric\par
\textbf{GFR} Gas Cooled Fast Reactor\par
\textbf{GIF} Generation IV International Forum\par
\textbf{GT-MHR} Gas Turbine Modular Helium Reactor\par
\textbf{HTR} High Temperature Reactor\par
\textbf{HTR-PM} High Temperature Reactor Pebble bed Modular\par
\textbf{HTTR} High Temperature Test Reactor\par
\textbf{IAEA} International Atomic Energy Agency\par
\textbf{ISIR} In Service Inspection and Repair\par
\textbf{JSFR} Japan atomic energy agency Sodium cooled Fast Reactor\par
\textbf{KALIMER} Korea Advanced Liquid MEtal Reactor\par
\textbf{LBE} Lead Bismuth Eutectic\par
\textbf{LFR} Lead Cooled Fast Reactor\par
\textbf{LWR} Light Water Reactor\par
\textbf{Magnox} Magnesium Non-Oxidising\par
\textbf{MOSART} MOlten Salt Actinide Recycler and Transmuter \par
\textbf{MSFR} Molten Salt Fast Reactor \par
\textbf{MSR} Molten Salt Reactor\par
\textbf{OPEC} Organization of the Petroleum Exporting Countries\par
\textbf{PRISM} Power Reactor Innovative Small Module\par
\textbf{PWR} Pressurised Water Reactor\par
\textbf{SCWR} Supercritical Water Cooled Reactor\par
\textbf{SFR} Sodium Cooled Fast Reactor\par
\textbf{SMR} Small Modular Reactor\par
\textbf{SSTAR} Small Secure Transportable Autonomous Reactor\par
\textbf{TRISO} Tri-isotropic\par
\textbf{VHTR} Very High Temperature Reactor\par
\textbf{VVER} Vodo-Vodyanoi Energetichesky Reactor \par


% An example of a floating figure using the graphicx package.
% Note that \label must occur AFTER (or within) \caption.
% For figures, \caption should occur after the \includegraphics.
% Note that IEEEtran v1.7 and later has special internal code that
% is designed to preserve the operation of \label within \caption
% even when the captionsoff option is in effect. However, because
% of issues like this, it may be the safest practice to put all your
% \label just after \caption rather than within \caption{}.
%
% Reminder: the "draftcls" or "draftclsnofoot", not "draft", class
% option should be used if it is desired that the figures are to be
% displayed while in draft mode.
%
%\begin{figure}[!t]
%\centering
%\includegraphics[width=2.5in]{myfigure}
% where an .eps filename suffix will be assumed under latex, 
% and a .pdf suffix will be assumed for pdflatex; or what has been declared
% via \DeclareGraphicsExtensions.
%\caption{Simulation Results}
%\label{fig_sim}
%\end{figure}

% Note that IEEE typically puts floats only at the top, even when this
% results in a large percentage of a column being occupied by floats.


% An example of a double column floating figure using two subfigures.
% (The subfig.sty package must be loaded for this to work.)
% The subfigure \label commands are set within each subfloat command, the
% \label for the overall figure must come after \caption.
% \hfil must be used as a separator to get equal spacing.
% The subfigure.sty package works much the same way, except \subfigure is
% used instead of \subfloat.
%
%\begin{figure*}[!t]
%\centerline{\subfloat[Case I]\includegraphics[width=2.5in]{subfigcase1}%
%\label{fig_first_case}}
%\hfil
%\subfloat[Case II]{\includegraphics[width=2.5in]{subfigcase2}%
%\label{fig_second_case}}}
%\caption{Simulation results}
%\label{fig_sim}
%\end{figure*}
%
% Note that often IEEE papers with subfigures do not employ subfigure
% captions (using the optional argument to \subfloat), but instead will
% reference/describe all of them (a), (b), etc., within the main caption.


% An example of a floating table. Note that, for IEEE style tables, the 
% \caption command should come BEFORE the table. Table text will default to
% \footnotesize as IEEE normally uses this smaller font for tables.
% The \label must come after \caption as always.
%
%\begin{table}[!t]
%% increase table row spacing, adjust to taste
%\renewcommand{\arraystretch}{1.3}
% if using array.sty, it might be a good idea to tweak the value of
% \extrarowheight as needed to properly center the text within the cells
%\caption{An Example of a Table}
%\label{table_example}
%\centering
%% Some packages, such as MDW tools, offer better commands for making tables
%% than the plain LaTeX2e tabular which is used here.
%\begin{tabular}{|c||c|}
%\hline
%One & Two\\
%\hline
%Three & Four\\
%\hline
%\end{tabular}
%\end{table}


% Note that IEEE does not put floats in the very first column - or typically
% anywhere on the first page for that matter. Also, in-text middle ("here")
% positioning is not used. Most IEEE journals use top floats exclusively.
% Note that, LaTeX2e, unlike IEEE journals, places footnotes above bottom
% floats. This can be corrected via the \fnbelowfloat command of the
% stfloats package.









% if have a single appendix:
%\appendix[Proof of the Zonklar Equations]
% or
%\appendix  % for no appendix heading
% do not use \section anymore after \appendix, only \section*
% is possibly needed

% use appendices with more than one appendix
% then use \section to start each appendix
% you must declare a \section before using any
% \subsection or using \label (\appendices by itself
% starts a section numbered zero.)
%





%\section{Proof of the First Zonklar Equation}
%Appendix one text goes here.

% you can choose not to have a title for an appendix
% if you want by leaving the argument blank
%\section{}
%Appendix two text goes here.


% use section* for acknowledgement
%\section*{Acknowledgment}


%The authors would like to thank...


% Can use something like this to put references on a page
% by themselves when using endfloat and the captionsoff option.
\ifCLASSOPTIONcaptionsoff
  \newpage
\fi



% trigger a \newpage just before the given reference
% number - used to balance the columns on the last page
% adjust value as needed - may need to be readjusted if
% the document is modified later
%\IEEEtriggeratref{8}
% The "triggered" command can be changed if desired:
%\IEEEtriggercmd{\enlargethispage{-5in}}

% references section

% can use a bibliography generated by BibTeX as a .bbl file
% BibTeX documentation can be easily obtained at:
% http://www.ctan.org/tex-archive/biblio/bibtex/contrib/doc/
% The IEEEtran BibTeX style support page is at:
% http://www.michaelshell.org/tex/ieeetran/bibtex/

%
% <OR> manually copy in the resultant .bbl file
% set second argument of \begin to the number of references
% (used to reserve space for the reference number labels box)
%\begin{thebibliography}{1}

%\bibitem{IEEEhowto:kopka}
%H.~Kopka and P.~W. Daly, \emph{A Guide to \LaTeX}, 3rd~ed.\hskip 1em plus
%  0.5em minus 0.4em\relax Harlow, England: Addison-Wesley, 1999.

% biography section
% 
% If you have an EPS/PDF photo (graphicx package needed) extra braces are
% needed around the contents of the optional argument to biography to prevent
% the LaTeX parser from getting confused when it sees the complicated
% \includegraphics command within an optional argument. (You could create
% your own custom macro containing the \includegraphics command to make things
% simpler here.)
%\begin{biography}[{\includegraphics[width=1in,height=1.25in,clip,keepaspectratio]{mshell}}]{Michael Shell}
% or if you just want to reserve a space for a photo:

%\begin{IEEEbiography}{Michael Shell}
%Biography text here.
%\end{IEEEbiography}

% if you will not have a photo at all:
%\begin{IEEEbiographynophoto}{John Doe}
%Biography text here.
%\end{IEEEbiographynophoto}

% insert where needed to balance the two columns on the last page with
% biographies
%\newpage

%\begin{IEEEbiographynophoto}{Jane Doe}
%Biography text here.
%\end{IEEEbiographynophoto}

% You can push biographies down or up by placing
% a \vfill before or after them. The appropriate
% use of \vfill depends on what kind of text is
% on the last page and whether or not the columns
% are being equalized.

%\vfill

% Can be used to pull up biographies so that the bottom of the last one
% is flush with the other column.
%\enlargethispage{-5in}



% that's all folks
\end{document}


